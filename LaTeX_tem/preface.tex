\chapter*{不用翻译的汉译 Strik0r 学术垃圾译丛 \quad 丛书前言}
\setcounter{page}{1}
\markboth{不用翻译的汉译 Strik0r 学术垃圾译丛}{丛书前言}
\addcontentsline{toc}{chapter}{丛书前言}
% 设置前言标题页的页码格式为empty, 即无页眉页脚
\thispagestyle{empty}

这套
\href{https://github.com/Strik0r-QF/Strik0r-s_GTM}{《不用翻译的汉译 Strik0r 学术垃圾译丛》}
(Strik0r's Garbage-like Texts in Mathematics/Computer Science/Physics, 以下简称 Strik0r's GTM)
又臭又长, 你只能选择性地阅读.
本丛书是我在大学期间学习到的知识和工作中的见闻和经验的总结 (说人话就是所有东西都是抄来的), 
内容横跨包括数学、物理学和计算机科学在内的多个学科及多个领域. 是一套纯粹的抄袭之作, 
经典的盗版, 标准的垃圾.

但就笔者不多的学习经验来说, \textbf{坚持输出倒逼输入}是笔者在这短短几年的学习生涯中总结出来的
一条关于学习的规律性认知, 这个认识与著名和 Feynmann 学习法、ZettelKasten 背后的原理和现代
脑科学和认知科学的相关学术研究成果不谋而合. 实践证明, 不断地通过创造各种形式的输出来检验
输入是否有效, 是确保学习成效和学习效率的关键因素和重要手段. 它有助于对知识点本身的理解, 也有助于
知识网络和知识体系的搭建, 可以达到 “既见树木, 又见森林” 的效果, 打破 “学而不思、思而不学” 的
怪圈. 因此这样一堆学术垃圾, 本质上是笔者学习、认识、实践和思考的副产物, 它们的诞生是有利于
笔者本人的学习的.

如今, 借由西北工业大学数学与统计学院开展的 “学霸直播间” 朋辈学习辅导交流活动的平台,
笔者的这些学术垃圾得以面世, 在学院内部和网络上经过两学期的使用后取得了不错的反响,
故笔者将这些学术垃圾整理成册, 最终形成了这套资料. 这套资料知识体系庞大、内容丰富、深入浅出,
均代表了笔者对该领域的最高了解程度 (也就是说, 书里有的我都会, 书里没有的我听都没听过).
主要包括了以下方面的内容:
\begin{enumerate}[label=$\left.\mathrm{\Alph*}\right)$, itemsep=0pt]
    \item \textbf{数学与自然科学类}: 主要包括数学分析 (或微积分, 高等数学)、
    高等代数 (或线性代数)、概率论与数理统计、
    解析几何、数学物理方法 (复变函数、积分变换、常微分方程、偏微分方程)、
    离散数学和抽象代数等数学课程和大学物理 (力学、振动与波动、波动光学、
    热学、电磁学、近代物理学部分)、电路分析等物理类课程的内容.
    \item \textbf{计算机科学类}: 主要包括程序设计基础 (Python、C、C++、Java、Kotlin)、
    软件技术 (MATLAB)、数据结构 (C、Python)、操作系统、计算机网络、计算机系统基础、
    计算机组成与体系结构等计算机科学与技术方向课程, 
    软件工程导论、面向对象分析与设计等软件工程方向课程,
    以及机器学习、神经网络、深度学习等人工智能方向课程的内容.
    \item \textbf{思想政治理论与军事类}: 主要包括马克思主义基本原理、中国近现代史纲要、
    思想道德与法治、毛泽东思想与中国特色社会主义理论体系概论、习近平新时代中国特色社会主义
    思想概论 5 门思政课程和军事理论、军事技能 2 门军事类课程的内容.
    \item \textbf{审美与艺术类}: 主要包括美学原理、中国审美历程课程的内容.
    后续考虑加入数学与音乐、基础乐理等音乐类的内容, 当前, 前提是我要进一步扩充我对有关方面
    的知识储备.
    \item \textbf{外国语言文学类}: 有一本专门命名为 English 的资料, 是我在本科期间所学
    英语课程的究极缝合怪.
    \item \textbf{运动与健康类}: 有一本运动与慢病防治, 说的就是这个内容.
    \item \textbf{写作与沟通类}: 沟通表达技能是非常重要且非常关键的, 因此我撰写了一系列与
    写作和沟通有关的文档, 归在这一类当中了, 现在正在写的有学术文档写作 (\LaTeX).
    \item \textbf{人文与社会科学、经济法律类}
\end{enumerate}

\subsection*{这套学术垃圾的编写原则}

在编写这套学术垃圾的时候, 我们遵循了以下原则:
\setcounter{subsubsection}{0}
\subsubsection{注意材料的普适性}

毕竟我只是个本科生, 没有任何一个高校会真的把我的资料当作教材来使用, 也不会有任何一个出版社愿意把
我的资料出版 (虽然我并没有联系过). 因此在编写资料的时候, 我尽可能地让它们具有普适性, 能够满足
大部分培养方案和教学大纲的学习要求, 以尽可能地让这些资料能够覆盖到更多的同学、帮助到更多的同学.

除此之外, 关于普适性的另一个阐述是同一份材料可以适应不同层次的课程教学需求, 例如《数学分析》可供
数学院系同学参考, 也可供修读高等数学或微积分课程的理工类、经管类院系的同学参考, 《程序设计基础》
系列可供各种专业院系的同学参考. 为了满足不同教学层次的需求, 我特地在每一本资料的前言中都列了一张
表格, 标注哪些内容是哪个层次的课程需要掌握的, 除此之外, 你也可以对照贵校的教学大纲以及相应课程
的考试大纲来确定你的学习范围.

把相似课程的内容全部结合在一起的另一个好处是, 可以很好地照顾到不同基础的同学. 大家来自全国各地,
起点的高低并不一致, 因此尽可能详细地介绍到更多的内容, 对于一些基础不是那么好的同学来说总归还是
好的. 此外, 严格按照课程大纲、考试大纲编写的教材、资料和参考书大家已经见的很多, 但详细、全面地
阐述一门学科到底在做什么的资料在国内是很罕见的.

\subsubsection{注重系列材料的统一性和整体性}

这个系列涉及的学科很多, 这个特点决定了我在编写这个系列的材料的时候会遇到很多来自不同领域的例子和
解决方案, 因此你在阅读的时候会遇到的一个比较常见的操作是: 在《计算机系统基础》或者类似学科的资料中
找到某一个理论, 然后在 《抽象代数》 中找到这个理论的分析或者证明 (放心, 我在行文的过程当中会给你
指路的), 如此一来便可以极大程度地减小我们的例题数量, 转而通过更多我们更容易接触到的实例来向大家展示
数学理论在工程实践和实际生活当中的应用. 相信对于兴趣广泛、求知欲强的同学来说, 一边阅读内容丰富、
取材广泛的材料, 一边巩固学习到的专业知识, 一边接触学科交叉领域当中的实际应用, 是一件非常享受的事.

\begin{flushright}
    \kaishu
    钱锋

    \includegraphics*[width = 20mm]{npu_2 2.png}
    \raisebox{0.5\height}{软件学院}

    2023 年 12 月
    \songti
\end{flushright}

\newpage
\thispagestyle{empty}